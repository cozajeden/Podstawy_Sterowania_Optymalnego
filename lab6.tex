\documentclass[12pt, letterpaper]{article}
\usepackage{import}
\usepackage[margin=0.5in]{geometry}
\usepackage{graphics}
\usepackage{xcolor}
\usepackage{bbm}
\usepackage[charter]{mathdesign}
\usepackage[hidelinks]{hyperref}


\graphicspath{{images/}}

\import{./preambule}{/python_highlights.tex}
\import{./preambule}{/polish.tex}
\import{./preambule}{/math_preambule.tex}
\import{./preambule}{/macros.tex}

\title{
    \huge Podstawy Sterowania Optymalnego\\Labolatorium 6\\
    \large Regulator LQR ze skończonym horyzontem czasowym
}
\author{Prowadzący: mgr inż. Krzysztof Hałas\\
        Wykonał: Ryszard Napierała}
\date{\today}

\begin{document}
    \maketitle

    \section*{Importy}
    \pythonl{lab6.py}{1}{7}

    \section*{Zadanie 2}
    \begin{enumerate}
        
        % 2.1
        \itemcl[]{Przygotować funkcję \emph{riccati(p,t)} implementująca różniczkowe równanie Riccatiego.
        Zdefiniować wektor chwil czasu od $t_1$ do $0$ przyjmując $t_1 = 5s$ Wykorzystując funkcję
        odeint wyznaczyć przebieg wartości macierzy $P$ w czasie. Zwrócić uwagę na konieczność
        konwersji macierzy $P$ do postaci wektorowej dla uzyskania zgodności z funkcją odeint.
        Wykorzystać na przykład np.reshape, squeeze oraz \emph{np.tolist.}}{
            lab6.py}{11}{77}

        % 2.2
        \itemcl[]{Wykreślić przebieg elementów macierzy $P(t)$ w czasie. Zweryfikować poprawność wyni-
        ków poprzez porównanie z warunkiem krańcowym.}{
            lab6.py}{80}{88}
            \outputTxt{snippets/lab6/zad2_2.txt}
            \outputImg{0.6}{images/lab6/zad2_2}

        % 2.3
        \itemcl[]{Przygotować funkcję \emph{model(x,t)} implementującą model dynamiki układu otwartego
        zgodnie z równaniem. Funkcja powinna przyjmować na wejściu stan układu $x$ oraz
        aktualną chwilę czasu $t$.}{
            lab6.py}{91}{99}

        % 2.4
        \itemcl[]{Zmodyfikować funkcję \emph{model(x,t)} tak, by wprowadzić do niej wyznaczone wcześniej
        wartości macierzy $P(t)$. Wykorzystać \emph{interpolate.interp1d} w celu określenia wartości
        macierzy $P(t)$ w wybranej chwili czasu.}{
            lab6.py}{102}{121}

        % 2.5
        \itemcl[]{Przeprowadzić symulację odpowiedzi obiektu na wymuszenie skokowe w czasie $t /4\in (0, 5)s$
        wykorzystując funkcję \emph{odeint}.}{
            lab6.py}{124}{130}
            \outputImg{0.6}{images/lab6/zad2_5}

        % 2.7
        \itemcl[]{Przeprowadzić symulację układu dla niezerowych warunków początkowych.
        Zbadać wpływ macierzy $S$, $Q$ oraz $R$ na przebieg odpowiedzi układu.}{
            lab6.py}{133}{211}

            \outputImg{1}{images/lab6/zad2_7}

            \emph{Czy macierze $S$, $Q$ oraz $R$ pozwalają dowolnie kształtować przebieg uchybu regula-
            cji? Czy istnieje jakaś zależność między doborem tych macierzy?}\\
            Macierze $Q, R, S$ nie pozwalają dowolnie kształtować przebiegu uchybu regulacji.
            Macierz $S$ jest zależna od macieży $Q$ oraz $S$.

        % 2.8
        \itemcl[]{Rozszerzyć funkcję \emph{model(x,t)} o wyznaczanie wartości wskaźnika jakości $J$. Funkcja
        \emph{model(x,t)} powinna wyznaczać pochodną (tj. wyrażenie podcałkowe) wskaźnika $J$ jako
        dodatkową zmienną stanu – zostanie ona scałkowana przez odeint, a jej wartość zwrócona
        po zakończeniu symulacji}{
            lab6.py}{214}{244}

            \outputTxt{snippets/lab6/zad2_7.txt}

            \emph{Czy wyznaczona wartość rzeczywiście odpowiada minimalizowanemu wyrażeniu?
            W jakim horyzoncie czasu została ona wyznaczona?}\\
            Wartość odpowiada minimalnemu wyrażeniu.\\
            Została ona wyznaczona w przedziale $t \in (0, 5)$.

        % 2.9
        \itemcl[]{Powtórzyć symulację dla $t_1 = 2s$ oraz zmiennych wartości nastaw $S$, $Q$, $R$.}{
            lab6.py}{247}{282}
            \outputImg{1}{images/lab6/zad2_9}

            \emph{Czy układ osiąga stan ustalony? Jaki teraz wpływ mają poszczególne nastawy?}\\
            W krótszym czasie układ nie osiąga stanu ustalonego.\\
            Wpływ nastaw jest identyczny jak w podpunkcie 6.
            
    \end{enumerate}

\end{document}