\documentclass[12pt, letterpaper]{article}
\usepackage{import}
\usepackage[margin=0.5in]{geometry}
\usepackage{graphics}
\usepackage{xcolor}
\usepackage{bbm}
\usepackage[charter]{mathdesign}
\usepackage[hidelinks]{hyperref}


\graphicspath{{images/}}

\import{./preambule}{/python_highlights.tex}
\import{./preambule}{/polish.tex}
\import{./preambule}{/math_preambule.tex}
\import{./preambule}{/macros.tex}

\title{
    \huge Podstawy Sterowania Optymalnego\\Labolatorium 8\\
    \large Modelowanie układów nieliniowych
}
\author{Prowadzący: mgr inż. Krzysztof Hałas\\
        Wykonał: Ryszard Napierała}
\date{\today}

\begin{document}
    \maketitle

    \section*{Zadanie 2}
    \begin{enumerate}
        
        % 2.1
        \itemcl[]{Zaimportować biblioteki \emph{numpy, scipy.integrate.odeint} oraz \emph{matplotlib.pyplot}.}{
            lab8_2.py}{2}{5}

        % 2.2
        \itemcl[]{Utworzyć model wyznaczający wartość $\frac{dy(t)}{dt}$, na podstawie równania.}{
            lab8_2.py}{8}{11}

        % 2.3
        \itemcl[]{Utworzyć wektor czasu $t\in(0,10)s$. Dla tak opisanych chwili czasowych wyznaczyć
        numerycznie rozwiązanie równania, dla zerowych warunków początkowych $(c = 0)$.
        Wykorzystać polecenie \emph{odeint}.}{
            lab8_2.py}{14}{16}

        % 2.4
        \itemcl[]{Utworzyć zmienną, która odpowiada analitycznemu rozwiązaniu. Porównać na
        wykresie rozwiązanie równania uzyskane numerycznie (\emph{odeint}) i analitycznie.}{
            lab8_2.py}{19}{27}
            \outputImg{0.6}{images/lab8/za_2_4}
            \emph{Czy rozwiązania się pokrywają? Jaki rodzaj numerycznego wyznaczania
            rozwiązania równania różniczkowego został użyty?}\\
            Wykresy się pokrywają\\
            Zostało użyte całkowanie układu równań różniczkowych zwyczajnych
            
    \end{enumerate}

    \section*{Zadanie 3}
    \begin{enumerate}
        
        % 3.1
        \itemcl[]{Zaimportować biblioteki \emph{numpy, scipy.integrate.odeint}
        oraz \emph{matplotlib.pyplot}.}{
            lab8_3.py}{2}{5}

        % 3.2
        \itemcl[]{Przyjąć następujące wartości zmiennych
        $k_p=2$, $\omega=4$, $\zeta=0.25$, $u(t)=\mathbbm{1}(t)$.}{
            lab8_3.py}{8}{13}

        % 3.3
        \itemcl[]{Utworzyć model wyznaczający wartość $\frac{d^2y(t)}{dt^2}$.}{
            lab8_3.py}{16}{27}

        % 3.4
        \itemcl[]{Utworzyć wektor czasu $t\in(0, 50)s$. Dla tak opisanych chwili czasowych
        wyznaczyć numerycznie rozwiązanie równania wykorzystując polecenie \emph{odeint}.}{
            lab8_3.py}{30}{32}

        % 3.5
        \itemcl[]{Wyświetlić rozwiązanie równania w funkcji czasu.}{
            lab8_3.py}{35}{40}
            \outputImg{0.6}{images/lab8/za_3_3}
            \emph{Jaki jest charakter odpowiedzi układu na wymuszenie skokowe?}\\
            Odpowiedź układu ma charakter liniowych gasnących oscylacji
            
    \end{enumerate}

    \section*{Zadanie 4}
    \begin{enumerate}
        
        % 4.1
        \itemcl[]{Przyjąć następujące wartości zmiennych $k_p=2$, $T=2$, $k_ob=4$, $x(t)=\mathbbm{1}(t)$.}{
            lab8_4.py}{6}{10}

        % 4.2
        \itemcl[]{Utworzyć funkcję \emph{feedback(t,y)} opisującą rozważany schemat blokowy.}{
            lab8_4.py}{13}{24}

        % 4.3
        \itemcl[]{Wyznaczyć odpowiedź układu zamkniętego przez zastosowanie funkcji odeint. Wykreślić
        jej przebieg w funkcji czasu.}{
            lab8_4.py}{27}{35}
            \outputImg{0.6}{images/lab8/za_4_3}

        % 4.4
        \itemcl[]{Zmieniać wartość sygnału zadanego $x(t)={1,2,3}*\mathbbm{1}(t)$ i obserwować
        stany ustalone odpowiedzi układu.}{
            lab8_4.py}{38}{47}
            \outputImg{0.6}{images/lab8/za_4_4}
            \emph{Czy zachowana jest zasada superpozycji i skalowania?}\\
            Zasada superpozycji i skalowania została zachowana.\\\\
            \emph{Czy układ ma charakter liniowy?}\\
            Układ ma charakter liniowy

        % 4.5
        \itemcl[]{Zmienić postać funkcji \emph{feedback(t,y)} tak, aby sygnał sterujący nie był ograniczony.
        Wyświetlić odpowiedzi układu dla różnych stałowartościowych sygnałów zadanych $x(t)$.}{
            lab8_4.py}{50}{71}
            \outputImg{0.6}{images/lab8/za_4_5}
            \emph{Czy układ ma charakter liniowy?}\\
            Układ ma charakter liniowy
            
    \end{enumerate}

    \section*{Zadanie 5*}
    \begin{enumerate}
        
        % 4.1
        \item \textbf{Przedstawić równanie opisujące wahadło w postaci dwóch
        równań pierwszego rzędu.}\\\\
        \[\ddot{\Theta}=\frac{1}{J}\tau_m
        -\frac{d}{J}\dot{\Theta}
        -\frac{mgR}{J}\sin(\Theta)\]
        \[x_1=\Theta\]
        \[\dot{x_1}=x_2\]
        \[\dot{x_2}=\frac{1}{J}\tau_m
        -\frac{d}{J}x_2
        -\frac{mgR}{J}\sin(x_1)\]

        % 4.2
        \itemcl[]{Utworzyć funkcję wyznaczającą $\frac{d^2}{dt^2}\Theta(t)$.
        Jako sygnał wejściowy przyjąć $\tau_m=A\cos(\omega t)$,
        gdzie $A=1.5$ a $\omega=0.65$. Wartość współczynnika tłumienia ustawić jako $d=0.5$.}{
            lab8_5.py}{7}{34}

        % 4.3
        \itemcl[]{Wyznaczyć i wykreślić rozwiązanie równania różniczkowego.}{
            lab8_5.py}{37}{51}
            \outputImg{0.6}{images/lab8/za_5_3}
            \emph{Jaki jest charakter odpowiedzi układu na sygnał zadany równy funkcji trygonometrycznej?}\\
            Odpowiedź ma charakter liniowy oscylacyjny.
            
    \end{enumerate}

\end{document}