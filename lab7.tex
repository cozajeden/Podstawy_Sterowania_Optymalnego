\documentclass[12pt, letterpaper]{article}
\usepackage{import}
\usepackage[margin=0.5in]{geometry}
\usepackage{graphics}
\usepackage{xcolor}
\usepackage{bbm}
\usepackage[charter]{mathdesign}
\usepackage[hidelinks]{hyperref}


\graphicspath{{images/}}

\import{./preambule}{/python_highlights.tex}
\import{./preambule}{/polish.tex}
\import{./preambule}{/math_preambule.tex}
\import{./preambule}{/macros.tex}

\title{
    \huge Podstawy Sterowania Optymalnego\\Labolatorium 7\\
    \large Sterowanie układami nieliniowymi przy pomocy metody SDRE
}
\author{Prowadzący: mgr inż. Krzysztof Hałas\\
        Wykonał: Ryszard Napierała}
\date{\today}

\begin{document}
    \maketitle

    \section*{Zadanie 4}
    \begin{enumerate}
        
        % 4.1
        \itemcl[]{Wyznaczyć macierze stanu dla wahadła z rysunku, a na następnie zaimplementować je do
        programu}{
            lab7.py}{9}{32}

        % 4.2
        \itemcl[]{Przygotować funkcję \emph{riccati(p,t)} Rozwiązującego równinie SDRE zarówno dla skończonego
        jak i nieskończonego horyzontu czasowego. W wariancie z skończonym
        horyzontem czasowym zdefiniować wektor chwil czasu od $t_1$ do $0$ przyjmując $t_1=5s$
        Wykorzystując funkcję odeint dla obu przebiegów wyznaczyć przebieg wartości macierzy
        $P$ w czasie. Dla nieskończonego horyzontu czasowego wykorzystać \emph{scipy.linalg.solve\_continuous\_are}.
        Zwrócić uwagę na konieczność konwersji macierzy $P$ do postaci
        wektorowej dla uzyskania zgodności z funkcją odeint. Wykorzystać na przykład
        \emph{np.reshape, squeeze() oraz np.tolist()}.}{
            lab7.py}{35}{100}
            \emph{Porównaj oba wyniki dla obu wariantów. Spróbuj zmienić początkowe wartości
            macierzy $P$, $Q$ oraz $R$ pozwalają dowolnie kształtować przebieg uchybu regulacji?
            Czy istnieje jakaś zależność między doborem tych macierzy?}\\
            Macierze $Q$ i $R$ pozwalają dowolnie kształtować przebieg uchybu regulacji\\
            Macierz $P$ jest zależna od $Q$ i $R$\\
            Macierz $Q$ określa funkcję kosztu dla zadanego uchybu regulacji\\
            Macierz $R$ określa funkcję kosztu dla zadanego sterowania

        % 4.3
        \itemcl[]{Wykreślić przebieg elementów macierzy $P(t)$ w czasie. Zweryfikować poprawność
        wyników poprzez porównanie z warunkiem krańcowym}{
            lab7.py}{103}{121}
            \outputTxt{snippets/lab7/zad4_3.txt}
            \outputImg{0.6}{images/lab7/zad4_3}
            $P$ dla nieskończonego horyzontu jest nieskończone\\
            $P$ dla skończonego horyzontu dąży do $-S$, nie udało mi się dojść do tego dlaczego

        % 4.4
        \itemcl[]{Przeprowadzić symulację odpowiedzi obiektu na wymuszenie skokowe w czasie $t\in(0, 5)s$
        wykorzystując funkcję \emph{odeint}.}{
            lab7.py}{124}{152}
            \outputImg{0.6}{images/lab7/zad4_4}

        % 4.5
        \itemcl[]{Przeprowadzić symulację układu dla niezerowych warunków początkowych.
        Zbadać wpływ macierzy $Q$ oraz $R$ na przebieg odpowiedzi układu.}{
            lab7.py}{155}{204}
            \outputImg{1}{images/lab7/zad4_5_1}
            \outputImg{1}{images/lab7/zad4_5_2}
            \emph{Czy macierze $Q$ oraz $R$ pozwalają dowolnie kształtować przebieg uchybu regulacji?
            Czy istnieje jakaś zależność między doborem tych macierzy?}\\
            Macierze $Q$ i $R$ pozwalają dowolnie kształtować przebieg uchybu regulacji\\
            Macierz $Q$ określa funkcję kosztu dla zadanego uchybu regulacji\\
            Macierz $R$ określa funkcję kosztu dla zadanego sterowania

        % 4.6
        \itemcl[]{Rozszerzyć funkcję \emph{model(x,t)} o wyznaczanie wartości wskaźnika jakości $J$. Funkcja
        \emph{model(x,t)} powinna wyznaczać pochodną (tj. wyrażenie podcałkowe) wskaźnika $J$ jako
        dodatkową zmienną stanu, zostanie ona scałkowana przez odeint, a jej wartość zwrócona
        po zakończeniu symulacji}{
            lab7.py}{207}{276}
            \outputImg{0.6}{images/lab7/zad4_6}

            \emph{Czy wyznaczona wartość rzeczywiście odpowiada minimalizowanemu wyrażeniu?
            W jakim horyzoncie czasu została ona wyznaczona?}\\
            Wyznaczona wartość $J$ odpowiada minimalnemu koszcie regulacji\\
            Została wyznaczona w czasie $t\in(0, 5)$

        % 4.7
        \itemcl[]{Powtórzyć symulację dla $t_1=2s$ oraz zmiennych wartości nastaw $Q$, $R$.}{
            lab7.py}{279}{291}
            \outputImg{1}{images/lab7/zad4_7_1}
            \outputImg{1}{images/lab7/zad4_7_2}

            \emph{Czy układ osiąga stan ustalony? Jaki teraz wpływ mają poszczególne nastawy?}\\
            Układ osiąga stan ustalony jedynie przy niektórych nastawach\\
            Nastawy mają identyczny wpływ jak w przypdaku symulacji dla $t_1=5s$
            
    \end{enumerate}

\end{document}