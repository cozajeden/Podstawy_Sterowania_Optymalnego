\documentclass[12pt, letterpaper]{article}
\usepackage{import}
\usepackage[margin=0.5in]{geometry}
\usepackage{graphics}
\usepackage{xcolor}
\usepackage{bbm}
\usepackage[charter]{mathdesign}
\usepackage[hidelinks]{hyperref}


\graphicspath{{images/}}

\import{./preambule}{/python_highlights.tex}
\import{./preambule}{/polish.tex}
\import{./preambule}{/math_preambule.tex}
\import{./preambule}{/macros.tex}

\title{
    \huge Podstawy Sterowania Optymalnego\\Labolatorium 7\\
    \large Sterowanie układami nieliniowymi przy pomocy metody SDRE
}
\author{Prowadzący: mgr inż. Krzysztof Hałas\\
        Wykonał: Ryszard Napierała}
\date{\today}

\begin{document}
    \maketitle

    \section*{Zadanie 4}
    \begin{enumerate}
        
        % 4.1
        \itemcl[]{Przygotować funkcję \emph{riccati(p,t)} implementująca różniczkowe równanie Riccatiego.
        Zdefiniować wektor chwil czasu od $t_1$ do $0$ przyjmując $t_1 = 5s$ Wykorzystując funkcję
        odeint wyznaczyć przebieg wartości macierzy $P$ w czasie. Zwrócić uwagę na konieczność
        konwersji macierzy $P$ do postaci wektorowej dla uzyskania zgodności z funkcją odeint.
        Wykorzystać na przykład np.reshape, squeeze oraz \emph{np.tolist.}}{
            lab7.py}{9}{32}

        % 4.2
        \itemcl[]{Wykreślić przebieg elementów macierzy $P(t)$ w czasie. Zweryfikować poprawność wyni-
        ków poprzez porównanie z warunkiem krańcowym.}{
            lab7.py}{35}{100}
            \outputTxt{snippets/lab6/zad2_2.txt}
            \outputImg{0.6}{images/lab6/zad2_2}
            % Macierze Q i R pozwalają dowolnie kształtować przebieg uchybu regulacji
            % Macierz P jest zależna od Q i R
            % Macierz Q określa funkcję kosztu dla zadanego uchybu regulacji
            % Macierz R określa funkcję kosztu dla zadanego sterowania

        % 4.3
        \itemcl[]{Przygotować funkcję \emph{model(x,t)} implementującą model dynamiki układu otwartego
        zgodnie z równaniem. Funkcja powinna przyjmować na wejściu stan układu $x$ oraz
        aktualną chwilę czasu $t$.}{
            lab7.py}{103}{121}
            % P dla nieskończonego horyzontu jest nieskończone
            % P dla skończonego horyzontu dąży do -S, nie udało mi się dojść do tego dlaczego

        % 4.4
        \itemcl[]{Zmodyfikować funkcję \emph{model(x,t)} tak, by wprowadzić do niej wyznaczone wcześniej
        wartości macierzy $P(t)$. Wykorzystać \emph{interpolate.interp1d} w celu określenia wartości
        macierzy $P(t)$ w wybranej chwili czasu.}{
            lab7.py}{124}{152}

        % 4.5
        \itemcl[]{Przeprowadzić symulację odpowiedzi obiektu na wymuszenie skokowe w czasie $t /4\in (0, 5)s$
        wykorzystując funkcję \emph{odeint}.}{
            lab7.py}{155}{204}
            \outputImg{0.6}{images/lab6/zad2_5}

            % Macierze Q i R pozwalają dowolnie kształtować przebieg uchybu regulacji
            % Macierz Q określa funkcję kosztu dla zadanego uchybu regulacji
            % Macierz R określa funkcję kosztu dla zadanego sterowania

        % 4.6
        \itemcl[]{Przeprowadzić symulację układu dla niezerowych warunków początkowych.
        Zbadać wpływ macierzy $S$, $Q$ oraz $R$ na przebieg odpowiedzi układu.}{
            lab7.py}{207}{276}

            \outputImg{1}{images/lab6/zad2_7}

            \emph{Czy macierze $S$, $Q$ oraz $R$ pozwalają dowolnie kształtować przebieg uchybu regula-
            cji? Czy istnieje jakaś zależność między doborem tych macierzy?}\\
            Macierze $Q, R, S$ nie pozwalają dowolnie kształtować przebiegu uchybu regulacji.
            Macierz $S$ jest zależna od macieży $Q$ oraz $S$.
        % Wyznaczona wartość J odpowiada minimalnemu koszcie regulacji
        % Została wyznaczona w czasie t \in [0, 5]

        % 4.7
        \itemcl[]{Przeprowadzić symulację układu dla niezerowych warunków początkowych.
        Zbadać wpływ macierzy $S$, $Q$ oraz $R$ na przebieg odpowiedzi układu.}{
            lab7.py}{279}{291}

            \outputImg{1}{images/lab6/zad2_7}

            \emph{Czy macierze $S$, $Q$ oraz $R$ pozwalają dowolnie kształtować przebieg uchybu regula-
            cji? Czy istnieje jakaś zależność między doborem tych macierzy?}\\
            Macierze $Q, R, S$ nie pozwalają dowolnie kształtować przebiegu uchybu regulacji.
            Macierz $S$ jest zależna od macieży $Q$ oraz $S$.
            
    \end{enumerate}

\end{document}