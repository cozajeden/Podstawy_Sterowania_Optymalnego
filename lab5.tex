\documentclass[12pt, letterpaper]{article}
\usepackage{import}
\usepackage[margin=0.5in]{geometry}
\usepackage{graphics}
\usepackage{xcolor}
\usepackage{bbm}
\usepackage[charter]{mathdesign}
\usepackage[hidelinks]{hyperref}


\graphicspath{{images/}}

\import{./preambule}{/python_highlights.tex}
\import{./preambule}{/polish.tex}
\import{./preambule}{/math_preambule.tex}
\import{./preambule}{/macros.tex}

\title{
    \huge Podstawy Sterowania Optymalnego\\Labolatorium 5\\
    \large Regulator LQR z nieskończonym horyzontem czasowym
}
\author{Prowadzący: mgr inż. Krzysztof Hałas\\
        Wykonał: Ryszard Napierała}
\date{\today}

\begin{document}
    \maketitle

    \section*{Zadanie 2}
    \begin{enumerate}
        
        \itemcl[]{Rozwiązać algebraiczne równanie Riccatiego dla dynamiki układu przyjmując
            jednostkową macierz $Q$ oraz $R=1$. Do znalezienia rozwiązania równania Riccatiego
            wykorzystać funkcję \emph{scipy.linalg.solve\_continuous\_are}. Wyznaczyć i wypisać wartości
            wzmocnień $K$.}{lab5.py}{23}{28}
            \outputTxt{snippets/lab5/zad2_1.txt}

        \itemcl[]{Przygotować funkcję model(x,t) implementującą model dynamiki układu otwartego
        zgodnie z równaniem. Funkcja powinna przyjmować na wejściu stan układu x oraz
        aktualną chwilę czasu t}{lab5.py}{31}{39}

        \itemcl[]{Przeprowadzić symulację odpowiedzi obiektu na wymuszenie skokowe w czasie $t \in (0,5)$ s
        wykorzystując funkcję odeint.}{lab5.py}{42}{52}
        \outputImg{0.6}{images/lab5/zad2_3}

        \itemcl[]{Zmodyfikować funkcję model(x,t) tak, by sygnał sterujący miał postać $u=-Kx$}{
            lab5.py}{55}{63}

        \itemcl[]{Przeprowadzić symulację układu dla niezerowych warunków początkowych. Zbadać
        wpływ macierzy $Q$ oraz $R$ na przebieg odpowiedzi układu.}{lab5.py}{66}{96}
        \outputImg{0.6}{images/lab5/zad2_5_1}
        \emph{Czy macierze $Q$ oraz $R$ pozwalają dowolnie kształtować przebieg uchybu regulacji?
        Czy istnieje jakaś zależność między doborem macierzy $Q$ oraz $R$?}\\
        Większe wartości $Q$ wydłużają czas regulacji i zmniejszają przeregulowanie, natomiast
        większe wartości $R$ skracają czas regulacji i zwiększają przeregulowanie.
        Co jest widoczne na poniższych przebiegach odpowiedzi skokowej układu.
        \outputImg{1}{images/lab5/zad2_5_2}

        \itemcl[]{Rozszerzyć funkcję model(x,t) o wyznaczanie wartości wskaźnika jakości J. Funkcja
        model(x,t) powinna wyznaczać pochodną (tj. wyrażenie podcałkowe) wskaźnika J jako
        dodatkową zmienną stanu – zostanie ona scałkowana przez odeint, a jej wartość zwrócona
        po zakończeniu symulacji.}{lab5.py}{99}{124}
        \outputImg{0.6}{images/lab5/zad2_6}
        \emph{Czy wyznaczona wartość rzeczywiście odpowiada minimalizowanemu wyrażeniu?
        W jakim horyzoncie czasu została ona wyznaczona?}\\
        Wartość odpowiada minimalnemu wyrażniu, została zaznaczona na powyższym wykresie.
    \end{enumerate}

    \section*{Zadanie 3}
    \begin{enumerate}
        \itemcl[]{Zmodyfikować funkcję model(x,t) tak, by sygnał sterujący wyznaczany był zgodnie ze
        schematem przedstawionym na rysunku. Wyznaczyć wzmocnienia regulatora zgodnie z
        algorytmem $LQR$}{lab5.py}{127}{155}
        \outputImg{0.6}{images/lab5/zad3_1}


        \itemcl[]{Przeprowadzić symulację układu zamkniętego dla wybranej wartości zadanej qd. Zbadać
        wpływ macierzy $Q$ oraz $R$ na przebieg odpowiedzi układu.}{lab5.py}{158}{185}
        Wpływ macierzy $Q$ i $R$ jest analogiczny jak w zadaniu 2, podpunkcie 5.
        \outputImg{1}{images/lab5/zad3_2}

    \end{enumerate}

\end{document}